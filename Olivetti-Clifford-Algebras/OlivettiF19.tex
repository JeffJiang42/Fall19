\documentclass[psamsfonts, 12pt]{amsart}
%
%-------Packages---------
%
\usepackage[h margin=1 in, v margin=1 in]{geometry}
\usepackage{amssymb,amsfonts}
\usepackage[all,arc]{xy}
\usepackage{tikz-cd}
\usepackage{enumerate}
\usepackage{mathrsfs}
\usepackage{amsthm}
\usepackage{mathpazo}
\usepackage{float}
\usepackage{slashbox}
%\usepackage{charter} %another font
%\usepackage{eulervm} %Vakil font
\usepackage{yfonts}
\usepackage{mathtools}
\usepackage{enumitem}
\usepackage{mathrsfs}
\usepackage{fourier-orns}
\usepackage[all]{xy}
\usepackage{hyperref}
\usepackage{url}
\usepackage{mathtools}
\usepackage{graphicx}
\usepackage{pdfsync}
\usepackage{mathdots}
\usepackage{calligra}
\usepackage{import}
\usepackage{xifthen}
\usepackage{pdfpages}
\usepackage{transparent}

\newcommand{\incfig}[2]{%
    \fontsize{48pt}{50pt}\selectfont
    \def\svgwidth{\columnwidth}
    \scalebox{#2}{\input{#1.pdf_tex}}
}
%
\usepackage{tgpagella}
\usepackage[T1]{fontenc}
%
\usepackage{listings}
\usepackage{color}

\definecolor{dkgreen}{rgb}{0,0.6,0}
\definecolor{gray}{rgb}{0.5,0.5,0.5}
\definecolor{mauve}{rgb}{0.58,0,0.82}

\lstset{frame=tb,
  language=Matlab,
  aboveskip=3mm,
  belowskip=3mm,
  showstringspaces=false,
  columns=flexible,
  basicstyle={\small\ttfamily},
  numbers=none,
  numberstyle=\tiny\color{gray},
  keywordstyle=\color{blue},
  commentstyle=\color{dkgreen},
  stringstyle=\color{mauve},
  breaklines=true,
  breakatwhitespace=true,
  tabsize=3
  }
%
%--------Theorem Environments--------
%
\newtheorem{thm}{Theorem}[section]
\newtheorem*{thm*}{Theorem}
\newtheorem{cor}[thm]{Corollary}
\newtheorem{prop}[thm]{Proposition}
\newtheorem{lem}[thm]{Lemma}
\newtheorem*{lem*}{Lemma}
\newtheorem{conj}[thm]{Conjecture}
\newtheorem*{quest*}{Question}
%
\theoremstyle{definition}
\newtheorem{defn}[thm]{Definition}
\newtheorem*{defn*}{Definition}
\newtheorem{defns}[thm]{Definitions}
\newtheorem{con}[thm]{Construction}
\newtheorem{exmp}[thm]{Example}
\newtheorem{exmps}[thm]{Examples}
\newtheorem{notn}[thm]{Notation}
\newtheorem{notns}[thm]{Notations}
\newtheorem{addm}[thm]{Addendum}
\newtheorem{exer}[thm]{Exercise}
%
\theoremstyle{remark}
\newtheorem{rem}[thm]{Remark}
\newtheorem*{claim}{Claim}
\newtheorem*{aside*}{Aside}
\newtheorem*{rem*}{Remark}
\newtheorem*{hint*}{Hint}
\newtheorem*{note}{Note}
\newtheorem{rems}[thm]{Remarks}
\newtheorem{warn}[thm]{Warning}
\newtheorem{sch}[thm]{Scholium}
%
%--------Macros--------
\renewcommand{\qedsymbol}{$\blacksquare$}
\renewcommand{\sl}{\mathfrak{sl}}
\newcommand{\Bord}{\mathsf{Bord}}
\renewcommand{\hom}{\mathsf{Hom}}
\renewcommand{\emptyset}{\varnothing}
\renewcommand{\O}{\mathscr{O}}
\newcommand{\R}{\mathbb{R}}
\newcommand{\ib}[1]{\textbf{\textit{#1}}}
\newcommand{\Q}{\mathbb{Q}}
\renewcommand{\H}{\mathbb{H}}
\newcommand{\Z}{\mathbb{Z}}
\newcommand{\N}{\mathbb{N}}
\newcommand{\C}{\mathbb{C}}
\newcommand{\A}{\mathbb{A}}
\newcommand{\F}{\mathbb{F}}
\newcommand{\M}{\mathcal{M}}
\newcommand{\dbar}{\overline{\partial}}
\newcommand{\zbar}{\overline{z}}
\renewcommand{\S}{\mathbb{S}}
\newcommand{\V}{\vec{v}}
\newcommand{\RP}{\mathbb{RP}}
\newcommand{\CP}{\mathbb{CP}}
\newcommand{\B}{\mathcal{B}}
\newcommand{\GL}{\mathrm{GL}}
\newcommand{\PGL}{\mathrm{PGL}}
\newcommand{\SL}{\mathrm{SL}}
\newcommand{\PSL}{\mathrm{PSL}}
\newcommand{\SP}{\mathrm{SP}}
\newcommand{\SO}{\mathrm{SO}}
\newcommand{\SU}{\mathrm{SU}}
\newcommand{\Cliff}{\mathrm{Cliff}}
\newcommand{\Pin}{\mathrm{Pin}}
\newcommand{\Spin}{\mathrm{Spin}}
\newcommand{\gl}{\mathfrak{gl}}
\newcommand{\g}{\mathfrak{g}}
\newcommand{\Bun}{\mathsf{Bun}}
\newcommand{\inv}{^{-1}}
\newcommand{\bra}[2]{ \left[ #1, #2 \right] }
\newcommand{\set}[1]{\left\lbrace #1 \right\rbrace}
\newcommand{\abs}[1]{\left\lvert#1\right\rvert}
\newcommand{\norm}[1]{\left\lVert#1\right\rVert}
\newcommand{\transv}{\mathrel{\text{\tpitchfork}}}
\newcommand{\defeq}{\vcentcolon=}
\newcommand{\enumbreak}{\ \\ \vspace{-\baselineskip}}
\let\oldexists\exists
\renewcommand\exists{\oldexists~}
\let\oldL\L
\renewcommand\L{\mathfrak{L}}
\makeatletter
\newcommand{\tpitchfork}{%
  \vbox{
    \baselineskip\z@skip
    \lineskip-.52ex
    \lineskiplimit\maxdimen
    \m@th
    \ialign{##\crcr\hidewidth\smash{$-$}\hidewidth\crcr$\pitchfork$\crcr}
  }%
}
\makeatother
\newcommand{\bd}{\partial}
\newcommand{\lang}{\begin{picture}(5,7)
\put(1.1,2.5){\rotatebox{45}{\line(1,0){6.0}}}
\put(1.1,2.5){\rotatebox{315}{\line(1,0){6.0}}}
\end{picture}}
\newcommand{\rang}{\begin{picture}(5,7)
\put(.1,2.5){\rotatebox{135}{\line(1,0){6.0}}}
\put(.1,2.5){\rotatebox{225}{\line(1,0){6.0}}}
\end{picture}}
\DeclareMathOperator{\id}{id}
\DeclareMathOperator{\im}{Im}
\DeclareMathOperator{\codim}{codim}
\DeclareMathOperator{\coker}{coker}
\DeclareMathOperator{\supp}{supp}
\DeclareMathOperator{\inter}{Int}
\DeclareMathOperator{\sign}{sign}
\DeclareMathOperator{\sgn}{sgn}
\DeclareMathOperator{\indx}{ind}
\DeclareMathOperator{\alt}{Alt}
\DeclareMathOperator{\Aut}{Aut}
\DeclareMathOperator{\trace}{trace}
\DeclareMathOperator{\ad}{ad}
\DeclareMathOperator{\End}{End}
\DeclareMathOperator{\Ad}{Ad}
\DeclareMathOperator{\Lie}{Lie}
\DeclareMathOperator{\spn}{span}
\DeclareMathOperator{\dv}{div}
\DeclareMathOperator{\grad}{grad}
\DeclareMathOperator{\Sym}{Sym}
\DeclareMathOperator{\sheafhom}{\mathscr{H}\text{\kern -3pt {\calligra\large om}}\,}
\newcommand*\myhrulefill{%
   \leavevmode\leaders\hrule depth-2pt height 2.4pt\hfill\kern0pt}
\newcommand\niceending[1]{%
  \begin{center}%
    \LARGE \myhrulefill \hspace{0.2cm} #1 \hspace{0.2cm} \myhrulefill%
  \end{center}}
\newcommand*\sectionend{\niceending{\decofourleft\decofourright}}
\newcommand*\subsectionend{\niceending{\decosix}}
\def\upint{\mathchoice%
    {\mkern13mu\overline{\vphantom{\intop}\mkern7mu}\mkern-20mu}%
    {\mkern7mu\overline{\vphantom{\intop}\mkern7mu}\mkern-14mu}%
    {\mkern7mu\overline{\vphantom{\intop}\mkern7mu}\mkern-14mu}%
    {\mkern7mu\overline{\vphantom{\intop}\mkern7mu}\mkern-14mu}%
  \int}
\def\lowint{\mkern3mu\underline{\vphantom{\intop}\mkern7mu}\mkern-10mu\int}
%
%--------Hypersetup--------
%
\hypersetup{
    colorlinks,
    citecolor=black,
    filecolor=black,
    linkcolor=blue,
    urlcolor=blacksquare
}
%
%--------Solution--------
%
\newenvironment{solution}
  {\begin{proof}[Solution]}
  {\end{proof}}
%
%--------Graphics--------
%
%\graphicspath{ {images/} }

\begin{document}
%
\author{Jeffrey Jiang}
%
\title{Clifford Algebras and Spin Groups}
%
\maketitle
%
One motivating reason for studying Clifford algebras and Spin groups is to better
understand the orthogonal groups $\mathrm{O}_n$ and special orthogonal groups $\SO_n$,
which consist of the transformations that preserve the standard inner product
(and orientation in the case of $\SO_n$) on $\R^n$. These groups are of physical
importance, since they are exactly the linear maps preserving our usual notions of
length and angle, which are derived from the standard inner product. \\

If you care about the representations of these groups, you might have noticed some
problems -- there exist representations of the Lie algebra $\mathfrak{so}_n$ that fail
the exponentiate to representations of $\SO_n$ (though they do exponentiate to
projective representations). The source of the problem lies in the topology of the
$\SO_n$ and $\mathrm{O}_n$.
%
\begin{thm}
For $n > 2$, we have
\[
\pi_1(SO_n) \cong \Z/2\Z
\]
\end{thm}
%
In the case $n=3$, this can be visualized using the plate/belt loop trick.
The failure of the orthogonal groups to be simply connected tells us that
there is information that we cannot recover from the Lie algebra. This motivates the
question :
%
\begin{quest*}
What is the universal cover of $\SO_n$? What is the representation theory of this
group?
\end{quest*}
%
You might already know the answer to this question in some simple cases.
%
\begin{exmp}
Consider the case of $SO_3$. If you are familiar with computer graphics, you might
know that we can represent rotations in $\R^3$ (i.e. elements of $SO_3$) by quaternions.
Let $\H$ denote the algebra of quaternions, which are elements of the form
\[
q = a + bi + cj + dk
\]
with $a,b,c,d \in \R$ and $i,j,k$ formal symbols satisfying the relations
\[
i^2 = j^2 = k^2 = ijk = -1
\]
embedding $\R^3$ into $\H$ via the mapping $(x,y,z) \mapsto xi + yj + zk$, we define
an action of the unit quaternions $\mathrm{Sp}(1) = S^3 \subset \H^\times$ on $\R^3$
(here $\mathrm{Sp}(1)$ denotes the compact symplectic group) by
\[
q\cdot v = qv\overline{q}
\]
where $\overline{q}$ is the quaternionic conjugate, i.e.
\[
\overline{a + bi + cj + dk} = a - bi - cj - dk
\]
Any orthogonal transformation $A \in SO_n$ can be represented by the action of some
$q \in S^3$. In addition $q$ and $-q$ define the same orthogonal transformation, which
tells us that $S^3$ is the universal cover of $SO_3$, and the covering map is
the quotient by antipodal points, which tells us that $SO_3 \cong \RP^3$.
\end{exmp}
%
The central ingredient to answering both of these questions stems from the classical
theorem
%
\begin{thm}[\ib{Cartan-Dieudonn\'e}]
Any orthogonal transformation $A \in \mathrm{O}_n$ can be expressed as the compositions
of at most $n$ reflections about hyperplanes.
\end{thm}
%
By a hyperplane, we mean a $n-1$ dimensional subspace $H \subset \R^n$, which can be
identified with a choice of a unit normal vector $v \in H^\perp$. Reflection about $H$
is then given by the
map $R_H : \R^n \to \R^n$ defined by
\[
R_H(w) = w - 2\langle w,v\rangle v
\]
The Cartan-Dieudonn\'e theorem is the key piece in understanding the Clifford algebra.
%
\begin{defn}
The Clifford algebra $\Cliff_n$ is the freest unital associative $\R$-algebra generated
by elements of $\R^n$ subject to  the relations
\begin{enumerate}
  \item $v^2 = -1$ for any unit vector $v \in \R^n \subset \Cliff_n$
  \item $vw = -wv$ for $v,w \in \R^n \subset \Cliff_n$ where $\langle v,w \rangle = 0$.
\end{enumerate}
\end{defn}
%
By ``generated by $\R^n$, we mean that every element in $\Cliff_n$ can be written as a
formal sum of formal products of vectors in $\R^n$. In particular, if we fix a basis for
$\R^n$, all elements in $\Cliff_n$ can be written as sums of formal products of these
basis elements. Let $\set{e_i}$ denote the standard orthonormal basis for $\R^n$ with
the standard inner product. Then the relations we specified tells us that the set
\[
\set{e_{i_1}\cdots e_{i_k} ~:~ 0 \leq k \leq n, 1 \leq i_1 < i_2 < \ldots < i_k \leq n}
\]
forms a basis for $\Cliff_n$, much like how they form a basis for the exterior algebra of
$\R^n$. Note that this implies that $\dim \Cliff_n = 2^n$. \\

Another equivalent definition of the Clifford algebra $\Cliff_n$ is via a universal
property.
%
\begin{defn}
The Clifford algebra $\Cliff_n$ is a unital associative equipped with a linear map
$i : \R^n \to \Cliff_n$ such that for any linear map $\varphi : V \to A$ into another
unital associative $\R$-algebra $A$ satisfying $(\varphi(v))^2 = -\langle v,v \rangle$,
there exists a unique map  $\widetilde{\varphi} : \Cliff_n \to A$ such that the diagram
\[\begin{tikzcd}
V \ar[d, "i"'] \ar[dr, "\varphi"]\\
\Cliff_n \ar[r, "\widetilde{\varphi}"'] & A
\end{tikzcd}\]
commutes.
\end{defn}
%
\begin{rem*}
Some choose the convention that unit vectors square to $1$ in the Clifford algebra, which
is one of the several places where people will have differing sign conventions.
\end{rem*}
%
What is the motivation for this construction? Inside $\Cliff_n$, we want vector
$v \in \R^n$ to represent a hyperplane reflection about $v^\perp$. The fact that
hyperplane reflections defined by orthogonal vectors commute explains the second
relation. Before going further, we prove a helpful formula.
%
\begin{lem}
Let $v,w \in \R^n \subset \Cliff_n$. Then $vw+wv = -2\langle v,w\rangle$
\end{lem}
%
\begin{proof}
Let $v = \sum_i v^ie_i$ and $w = \sum_j w^je_j$. We compute
\begin{align*}
vw+wv &= \sum_{i,j} (v^iw^je_ie_j + v^iw^je_je_i) \\
&= \sum_{i,j} v^iw^j(e_ie_j + e_je_i)
\end{align*}
We note that since $\langle e_i, e_j \rangle = 0$, we have that $e_ie_j = -e_je_i$. This,
along with the fact that $e_i^2 = -1$ gives us that this sum becomes
\[
\sum_i v^iw^i(e_i^2+e_i^2) = -2\sum_i v^iw^i = -2\langle v,w \rangle
\]
\end{proof}
%
Let $\alpha : \Cliff_n \to \Cliff_n$ be the automorphism extending the map
$v \mapsto -v$ on $\R^n \subset \Cliff_n$ (i.e. for $v_1,\ldots v_k \in \R^n$,
$\alpha(v_1\cdot v_k) = (-1)^kv_1\cdots v_k$.) and let $G \subset \Cliff_n^\times$
denote the subgroup of the group of units generated by taking products of unit vectors
in $\R^n$. We then define a group action of $G$ on $\R^n \subset \Cliff_n$ by the formula
\[
g \cdot v = \alpha(g)vg\inv
\]
(Note the similarity of this action to the one of the quaternions on $R^3$.)
We first need to verify that this defines a group action, i.e. that it maps $\R^n$ back
to itself. To show this, it suffices to check on the generating set of unit vectors. Let
$v \in \R^n$ be a unit vector. We note that $v\inv = -v$, and $\alpha(v) = -v$. We then
compute for $w \in \R^n$
\begin{align*}
v \cdot w &= \alpha(v)wv\inv \\
&= vwv \\
&= (-2\langle v,w\rangle -wv)v \\
&= -2\langle v,w\rangle v + w
\end{align*}
%
Note that this just the definition of hyperplane reflection about $v^\perp$! Therefore,
not only do we have a group action, but we also have that $G$ acts on $\R^n$ by
orthogonal transformations, giving us a map $G \to O_n$. This map is surjective by the
Cartan-Dieudonn\'e theorem, and some additional work shows that the kernel of this map
is $\set{\pm 1}$, giving us a $2-1$ map. This tells us that $G$ is the double cover of
$\mathrm{O}_n$ that we are looking for. The group $G$ called the \ib{Pin group}
$\Pin_n$. Other thing to note is that a single hyperplane reflection is orientation
reversing, so the composition of an even number of reflections is orientation
preserving. Restricting to even products of unit vectors gives us the \ib{Spin group}
$\Spin_n$, which is the simply connected double cover of $\SO_n$. \\
%

To investigate these groups further, we study the structure of the Clifford algebras
they came from. The first goal is to identify the Clifford algebras in terms of
algebras we understand well (e.g. matrix algebras). It's not too hard to
identify the first few
%
\begin{exmp} \enumbreak
\begin{enumerate}
  \item $\Cliff_0 \cong \R$
  \item $\Cliff_1 \cong \C$ via the map $e_1 \mapsto i$.
  \item $\Cliff_2 \cong \H$ via the map $e_1 \mapsto i$, $e_2 \mapsto j$.
\end{enumerate}
\end{exmp}
%
The name of the game here is to find algebras with an anti-commuting basis of elements
that square to $-1$. For the low dimensional cases, this is easy, but things will
quickly become difficult, since $\dim \Cliff_n = 2^n$. To find some more
(in fact, to find them all), we develop a little more technology.
%
\begin{defn}
Let $A$ be an associative $\R$-algebra. A $\Z/2\Z$-grading on $A$ is a direct sum
decomposition
\[
A = A^0 \oplus A^1
\]
such that multiplication respects the grading, i.e. for $a \in A^i$ and $b \in A^j$,
$ab \in A^{i+j \mod 2}$. The subspaces $A^0$ and $A^1$ are called \ib{even} and
\ib{odd} respectively. Note in particular that $A^0$ forms a subalgebra of $A$, since
the product of even elements is even. The elements of $A^0$ and $A^1$ are called
\ib{homogeneous elements}. For a homogeneous element $a$, we let $|a| \in \set{0,1}$
denote the index of subspace it lies in.
\end{defn}
%
\begin{exmp}\enumbreak
\begin{enumerate}
  \item For a vector space $V$, $\Lambda^\bullet V$ has a $\Z/2\Z$-grading
  \[
  \Lambda^\bullet V = \Lambda^{\text{even}}V \oplus \Lambda^{\text{odd}}V
  \]
  \item The Clifford algebra has a $\Z/2\Z$-grading in much the same way, where
  $\Cliff^0_n$ is the algebra of even products of basis vectors, and $\Cliff^1_n$ is the
  subspace of odd products of basis vectors.
\end{enumerate}
\end{exmp}
%
Something that will come in handy later :
%
\begin{thm}
For $n \geq 1$, the even subalgebra $\Cliff^0_n \subset \Cliff_n$ is isomorphic to
$\Cliff_{n-1}$ as ungraded algebras.
\end{thm}
%
\begin{proof}
We note that for the standard basis $\set{e_i}$ for $\R^n$, we have that
\[
(e_ie_j)^2 = e_ie_je_ie_j = -e_i^2e_j^2 = -1
\]
and that a generating set for $\Cliff^0_n$ is
\[
\set{e_1e_i ~:~ 2 \leq i \leq n}
\]
The mapping $e_1e_i \mapsto e_{i-2}$ defines the desired isomorphism.
\end{proof}
%
Another helpful operation is taking tensor products. Since things are graded, we
need a small modification of the definition.
%
\begin{defn}
Let $A$ and $B$ be $\Z/2\Z$-graded algebras. Define the \ib{graded tensor product} of
$A$ and $B$, denoted $A \otimes B$ to be the algebra with the underlying vector space
$A \otimes B$ with the multiplication defined on homogeneous elements by
\[
(a_1\otimes b_1)(a_2\otimes b_2) = (-1)^{|b_1||a_2|}a_1a_2\otimes b_1b_2
\]
This algebra is once again graded, where
\begin{align*}
(A \otimes B)^0 = (A^0 \otimes B^0) \oplus (A^1 \otimes B^1) \\
(A \otimes B)^1 = (A^0 \otimes B^1) \oplus (A^1 \otimes B^0)
\end{align*}
\end{defn}
%
\begin{rem*}
A heuristic reason for this definition is that when we are multiplying, we are formally
``commuting $b_1$ with $a_2$, so we should pick up an extra sign depending on their
parities.
\end{rem*}
%
The graded tensor product allows us to express higher dimensional Clifford algebras
in terms of their smaller brethren.
%
\begin{thm}
\[
\Cliff_p \otimes \Cliff_q = \Cliff_{p+q}
\]
where $\otimes$ denotes the graded tensor product.
\end{thm}
%
\begin{proof}
We specify an isomorphism $\varphi : \Cliff_{p+q} \to \Cliff_p \otimes \Cliff_q$
on an orthonormal basis and verifying that it satisfies the Clifford relation. Let
$\set{e_i}$ and $\set{f_i}$ be the standard orthonormal bases for $\R^p$ and $\R^q$
respectively, and let $\set{b_i}$ denote the standard orthonormal basis for $\R^{p+q}$
\[
\varphi(b_i) = \begin{cases}
e_i \otimes 1 & 1 \leq i \leq p \\
1 \otimes f_i & p+1 \leq i \leq p+q
\end{cases}
\]
to show the Clifford relations are satisfied, we just need to verify that for every $i$,
we have $(\varphi(b_i))^2 = -1$, and for distinct $i,j$, we have that
$\varphi(b_i)$ and $\varphi(b_j)$ anticommute. The first condition is obvious. For
the second, there are two cases. If $1 \leq i,j \leq p$, they anticommute because the
$e_i$ anticommute. The same holds for $p \leq i,j \leq p+q$. In the case that
$1 \leq i \leq p$ and $p+1 \leq j \leq p+q$, we compute
\begin{align*}
\varphi(b_i)\varphi(b_j) + \varphi(b_j)\varphi(b_i) &= (e_i \otimes 1)(1 \otimes f_j)
+ (1 \otimes f_j)(e_i \otimes 1) \\
&= e_i \otimes f_j - e_i \otimes f_j = 0
\end{align*}
where we use the fact that $1$ is an even element and the fact that $f_j$ and $e_i$
are odd.
\end{proof}
%
This fact, along with some computations by hand gives us that the first
$9$ Clifford algebras (starting at $\Cliff_0$) are isomorphic as ungraded algebras
to the following list.
\begin{align*}
\R \qquad \C \qquad \H \qquad \H \times \H \qquad M_2\H \qquad M_4\C \qquad M_8\R
\qquad M_8\R \times M_8\R \qquad M_{16}\R
\end{align*}
%
Beyond this, something curious happens : the graded tensor product with $M_{16}\R$
is the same as the ungraded tensor product on the level of ungraded algebras, so
these $9$ are the only ones we need to classify them all. In particular, we know
that all Clifford algebras are very nice -- they are matrix algebras over
$\R$, $\C$, or $\H$. In particular, this implies that they are semisimple algebras,
which tells us that they have nice modules. In particular, this implies that all Clifford modules
are direct sums of irreducible ones, and from these irreducible modules, we can recover
the lost representations of $\SO_n$.
%
\begin{defn}
The \ib{Spin representations} are the representations of $\Spin_n$ obtained by
restricting the action of $\Cliff_n$ on an irredicible module.
\end{defn}
%
The classification of these representations is surprisingly simple, which stems
from two observations.
%
\begin{thm}
The even subalgebra $\Cliff^0_n$ is isomorphic as ungraded algebras with $\Cliff_{n-1}$.
\end{thm}
%
\begin{thm}
Let $k = \R$, $\C$, or $\H$.
\begin{enumerate}
  \item The algebra $M_nk$ admits a single irreducible left
  module, which is $M_nk$ acting on $k^n$ in the standard way.
  \item The algebra $M_nk \times M_nk$ admits two irreducible modules, corresponding
  to the left or right factor acting in the standard way, and the other acting by $0$.
\end{enumerate}
\end{thm}
%
\begin{proof} \enumbreak
\begin{enumerate}
  \item We note that $M_nk$ acts transitively on $k^n$, so $k^n$ is an irreducible module. To
  prove that it is unique, we note that $M_nk$ admits an increasing chain of left ideals
  \[
  0 = I_0 \subset I_1 \subset \ldots \subset I_n = M_nk
  \]
  where $I_k$ is the left ideal of matrices where all of the entries but the first $k$ columns
  are $0$. These ideals have the property that the quotients $I_k/I_{k-1}$ are all isomorphic
  to $k^n$ as $M_nk$ modules. Let $M$ be an arbitrary nontrivial irreducible module, and fix
  $m \in M$. Since $M$ is trivial, the orbit of $m$ under the action of $M_nk$ must necessarily
  be all of $M_nk$. We then get a surjective module homomorphism $\varphi : M_nk \to M$
  where $A \mapsto A\cdot m$. Therefore, there exists some smallest $k$ such that $\varphi(I_k)$
  is nonzero, and by construction, $\varphi\vert_{I_k}$ factros through $I_k/I_{k-1}$, which
  is necessarily an isomorphism by Schur's Lemma.
  \item The proof is much the same, where we use the increasing chain
  \[
  0 = J_0 \subset I_1 \times \set{0} \subset \ldots \subset M_nk \times \set{0} \subset M_nk
  \times I_1 \subset \ldots \subset M_nk \times M_nk
  \]
  where $I_k$ is defined the same as in the first case. The proof is then the same argument,
  using the observation that $J_k/J_{k-1}$ is isomorphic to $k^n$ with the standard action
  of the left factor or the right factor.
\end{enumerate}
\end{proof}
%
Having classified ungraded Clifford modules, this tells us the Spin representations. For
example, in the case of $\Spin_3 \cong SU_2 \cong Sp(1)$, we have that the Spin representation
is just the action of the Spin group on the irreducible module of the even subalgebra
$\Cliff_3^0$, which is isomorphic to $\Cliff_2 \cong \H$. So the Spin representation is just
$Sp(1)$ acting on $\H$ in the standard way. \\

We discuss one final thing about Clifford modules. For a ring $A$, let ${}_A\mathsf{Mod}$
denote commutative monoid of left $A$-modules under direct sum. Given a ring homomorphism
$\varphi : A \to B$, there is an induced map ${}_B \mathsf{Mod} \to {}_A \mathsf{Mod}$,
where a $B$ module $M$ becomes an $A$-module with multiplication defined by
$a \cdot m = \varphi(a) \cdot m$. Studying this induced map on modules proves extremely fruitful
for Clifford modules. let $\mathcal{M}_n$ denote the commutative monoid of left $\Cliff_n$
modules. The inclusion maps $\R^n \hookrightarrow \R^{n+1}$ induces an inclusion
$\Cliff_n \hookrightarrow \Cliff_{n+1}$, which then gives maps
$\mathcal{M}_{n+1} \to \mathcal{M}_n$. Since the maps $\Cliff_n \hookrightarrow \Cliff_{n+1}$
are inclusion, the induce map on $\mathcal{M}_n$ is just the restriction of a module of
$\Cliff_{n+1}$ to the action of $\Cliff_n \subset \Cliff_{n+1}$. We can then compute the
cokernel of this map, which computes to obstruction to extending a $\Cliff_n$-module to a
$\Cliff_{n+1}$ module. To compute these cokernels, we note that the inclusions
$\Cliff_n \hookrightarrow \Cliff_{n+1}$ fall into one of the following cases
\begin{enumerate}
  \item $M_nk \hookrightarrow M_nk'$ where $k = \R$ or $\C$, and $k' = \C$ or $\H$ is the
  division algebra of twice the dimension over $\R$.
  \item $M_nk \times M_nk \hookrightarrow M_{2n}k$, given by the mapping
  \[
  (A,B) \mapsto \begin{pmatrix}
  A & 0 \\
  0 & B
  \end{pmatrix}
  \]
  \item $M_nk' \hookrightarrow M_{2n}k$, where $k$ and $k'$ are defined as in (1)
  \item $M_nk \hookrightarrow M_nk \times M_nk$
\end{enumerate}
%
Using our classification of Clifford modules, as well as their semisimplicty, we compute the
cokernels in all these cases
%semisimple
\begin{enumerate}
  \item In this case, the irreuducible module for $M_nk'$ is $(k')^n$, which is twice the
  dimension of $k^n$ over $\R$. Since both $M_nk$ and $M_nk'$ only admit a single
  irreducible module, the induced monoid homomorphism is the map
  $\Z^{\geq 0} \to \Z^{\geq 0}$ where $1 \mapsto 2$. The cokernel is the \emph{group}
  $\Z/2\Z$.
  \item The irreducible modules $k^{2n}$ decomposes into a direct sum of the two
  irreducible modules for $M_nk \times M_nk$. The monoid homomorphism is then the map
  $\Z^{\geq 0} \to \Z^{\geq 0}$ defined by $1 \mapsto (1,1)$. The cokernel is then the
  group $\Z$. You might have seen this construction before as the
  \ib{Grothendieck completion} of a commutative monoid.
  \item The irreducible modules $(k')^n$ and $k^{2n}$ are the same dimension so the
  monoid homomorphism is just $1\mapsto 1$. so the cokernel is trivial.
  \item Again, both irreducible modules for $M_nk \times M_nk$ are the same dimension as
  the one for $M_nk$, so the map is given by $(1,0) \mapsto 1$ and $(0,1)\mapsto 1$, so
  the cokernel is again trivial.
\end{enumerate}
%
Therefore, these cokernels give the following $8$-periodic sequence
\[
\Z/2\Z \qquad \Z/2\Z \qquad 0 \qquad \Z \qquad 0 \qquad 0 \qquad 0 \qquad \Z
\]
which is affectionately called the \ib{Bott song} (though perhaps shifted by 1), and is
one of the many avatars of \ib{Bott periodicity}.
%
\end{document}